\documentclass{article}
\usepackage[UTF8]{ctex}
% \usepackage[showframe]{geometry} %调整页边距showframe显示框架
\usepackage{amsmath}  %数学环境
\usepackage{paralist,bbding,pifont} %罗列环境
\usepackage{lmodern}  %中文环境与amsmath格式冲突
\usepackage{array,graphicx}  %插入表格、图片
\usepackage{booktabs}
\usepackage{float}
\usepackage{appendix}
\usepackage{amssymb}
\usepackage{amsthm}
\usepackage{tocloft}  %目录
\usepackage{listings}
\usepackage{xcolor}
\usepackage{hyperref}
\usepackage{setspace}
\usepackage{algorithm}
\usepackage{algpseudocode}
\renewcommand\cftsecdotsep{\cftdotsep}
\renewcommand\cftsecleader{\cftdotfill{\cftsecdotsep}}
\renewcommand {\cftdot}{$ \cdot $}
\renewcommand {\cftdotsep}{1.5}
\hypersetup{colorlinks=true,linkcolor=black}
\usepackage[a4paper, portrait, margin=2.5cm]{geometry}
\renewcommand{\baselinestretch}{1.25} %行间距取多倍行距(设置值为1.5)
\setlength{\baselineskip}{20pt} 

\usepackage{listings}%插入代码
\usepackage{color}
\lstset{%代码格式的配置
extendedchars=false,            % Shutdown no-ASCII compatible
language=Matlab,                % !!!选择代码的语言
basicstyle=\footnotesize\tt,    % the size of the fonts that are used for the code
tabsize=3,                            % sets default tabsize to 3 spaces
numbers=left,                   % where to put the line-numbers
numberstyle=\tiny,              % the size of the fonts that are used for the line-numbers
stepnumber=1,                   % the step between two line-numbers. If it's 1 each line
                                % will be numbered
numbersep=5pt,                  % how far the line-numbers are from the code   %
keywordstyle=\color[rgb]{0,0,1},                % keywords
commentstyle=\color[rgb]{0.133,0.545,0.133},    % comments
stringstyle=\color[rgb]{0.627,0.126,0.941},      % strings
backgroundcolor=\color{white}, % choose the background color. You must add \usepackage{color}
showspaces=false,               % show spaces adding particular underscores
showstringspaces=false,         % underline spaces within strings
showtabs=false,                 % show tabs within strings adding particular underscores
frame=single,                   % adds a frame around the code
captionpos=b,                   % sets the caption-position to bottom
breaklines=true,                % sets automatic line breaking
breakatwhitespace=false,        % sets if automatic breaks should only happen at whitespace
title=\lstname,                 % show the filename of files included with \lstinputlisting;
                                % also try caption instead of title
mathescape=true,escapechar=?    % escape to latex with ?..?
escapeinside={\%*}{*)},         % if you want to add a comment within your code
%columns=fixed,                  % nice spacing
%morestring=[m]',                % strings
%morekeywords={%,...},%          % if you want to add more keywords to the set
%    break,case,catch,continue,elseif,else,end,for,function,global,%
%    if,otherwise,persistent,return,switch,try,while,...},%
 }
%% 页眉
\usepackage{fancyhdr}
\newcommand{\myname}{李佳}
\newcommand{\myid}{2100010793}
\pagestyle{fancy}
\fancyhf{}
\rhead{\myid}
\lhead{\myname}
\cfoot{\thepage}

%%%% Declare %%%
\DeclareMathOperator{\Ran}{Ran}
\DeclareMathOperator{\Dom}{Dom}
\DeclareMathOperator{\Rank}{Rank}

\newcommand{\md}{\mathrm{d}}
\newcommand{\mR}{\mathbb{R}}
\newcommand{\mbF}{\mathbb{F}}
%%% Declare %%%
\newtheorem{innercustomthm}{Problem}
\newenvironment{prob}[1]
{\renewcommand\theinnercustomthm{#1}\innercustomthm}
{\endinnercustomthm}

%%设置
\title{数值代数$\ $第3次上机作业}
\author{李佳~2100010793}
\date{}

%%正文

\begin{document}
\zihao{-4}
\maketitle
\begin{section}{问题描述}
    \noindent(1) 估计5-20阶Hilbert矩阵的$\infty$范数条件数.
    
    \noindent(2) 设$$A_n = \begin{bmatrix}
        1 & 0 & \dots & 0 & 1\\
        -1 &\ddots &\ddots & \vdots &\vdots\\
        \vdots &\ddots & \ddots & 0 & 1\\
        -1&\dots & -1 & 1 & 1\\
        -1&\dots & -1 & -1 & 1
    \end{bmatrix}\in \mathbb{R}^{n\times n}.$$
    先随机地选取$x\in\mathbb{R}^n,$并计算出$b=A_nx$;然后再利用列主元Guass消去法求解
    该方程组,假定计算解为$\hat{x}$.试对$n$从5到30估计计算解$\hat{x}$的精度,并且与真实相对
    误差作比较.
\end{section}
\begin{section}{数值方法}
    \begin{subsection}{估计矩阵的1-范数:优化法}
        为估计矩阵$A$的$\infty$范数下条件数,需要估计$A^{-1}$的$\infty$范数,也就是$A^{-T}$的1范数.
        通过教材上的定理 2.5.1 知,要估计矩阵$B$的1范数,可以选定向量$x$满足$\lVert x\rVert_1=1$且$Bx$的各分量非0,计算$sign(Bx)=v$,$z=B^Tv$.
        若$\lVert z\rVert_{\infty}=z^Tx$,则$\lVert Bx\rVert_1$取得最大值,从而$\lVert B\rVert_1=\lVert Bx\rVert_1$;
        若$\lVert z\rVert_{\infty}>z^Tx$,则$\lVert Be_j\rVert_1>\lVert Bx\rVert_1,$其中$j$满足$z_j = \lVert z\rVert_{\infty}$,
        取$x=e_j$再回到第一步进行迭代即可.

        而在计算$A^{-T}$的1范数时,需要计算的$w=Bx,z=B^Tv$就相当于解方程$A^Tw=x,Az=v,$可利用先前通过列主元Guass消去法得到的$A$的三角分解
        进行计算.
    \end{subsection}
    \begin{subsection}{估计计算解的相对误差}
        教材2.5.1给出了计算解相对误差的估计:
        $$ \dfrac{\lVert x-\hat{x}\rVert_{\infty}}{\lVert x\rVert_{\infty}} \leq \lVert A\rVert_{\infty}\lVert A^{-1}\rVert_{\infty}\dfrac{\lVert r\rVert_{\infty}}{\lVert b\rVert_{\infty}},$$
        其中$$r=b-A\hat{x}.$$
        
        $\lVert A\rVert_{\infty}$通过直接计算每行元素绝对值之和的最大值得到;$\ \lVert A^{-1}\rVert_{\infty}$通过2.1节的方法进行估计;
        $\lVert r\rVert_{\infty},\lVert b\rVert_{\infty}$通过计算向量各分量绝对值最大值得到.
        
    \end{subsection}
\end{section}
\begin{section}{理论分析结果}
    \noindent(1)考虑比对$n$阶Hilbert矩阵的最后两行:
    $$ (\dfrac{1}{n-1},\dfrac{1}{n},...,\dfrac{1}{2n-2}),$$
    $$ (\dfrac{1}{n},\dfrac{1}{n+1},...,\dfrac{1}{2n-1}).$$
    这两行几乎是线性相关的,大概可知随着阶数增加,Hilbert矩阵越接近奇异矩阵,其条件数也可能越大.

    \noindent(2)随着矩阵阶数增加,列主元高斯消去法及求解过程带来的舍入误差增大,因此真实的求解误差
    可能随阶数增加而增大; 随着矩阵阶数增加,矩阵的条件数求解误差的估计也可能随阶数增加而增大.
\end{section}

\begin{section}{具体算法实现}
    \begin{subsection}{估计矩阵的条件数}
        \begin{lstlisting}
function [norm1] = InftyNorm_Inv(A)
OK = 1; % 用于判定是否终止循环
[~,n] = size(A);
[A0,u] = LU_col(A);        % 提前作列主元三角分解
x = ones(n,1)/n; B0 = A0'; % 定义初始向量 对矩阵作转置
while OK == 1 
    w = x;  
    % 求解 A^Tw=x
    for i = 1:n-1  % 前代法解 U^Ty=b(y 记录在 b 中)
        w(i) = w(i)/B0(i,i);
        w(i+1:n) = w(i+1:n) - w(i)* B0(i+1:n,i);
    end
    w(n) = w(n)/B0(n,n);
    for j = n:-1:2 % 回代法解 L^Tx=y(x 记录在 b 中)
        w(1:j-1) = w(1:j-1) - w(j)* B0(1:j-1,j);
    end
    for k = n-1:-1:1  % 向量进行行交换(计算P^Tb)
        mid = w(k);
        w(k) = w(u(k));
        w(u(k)) = mid;
    end
    z = sign(w);
    % 求解 Az=v
    for k = 1:n-1  % 向量进行行交换(计算P^(-1)b)
        mid = z(k);
        z(k) = z(u(k));
        z(u(k)) = mid;
    end
    for i = 1:n-1  % 前代法解 Ly=z(y 记录在 z 中)
        z(i+1:n) = z(i+1:n) - z(i)* A0(i+1:n,i);
    end
    for j = n:-1:2 % 回代法解 Ux=y(x 记录在 z 中)
        z(j) = z(j)/A0(j,j);
        z(1:j-1) = z(1:j-1) - z(j)* A0(1:j-1,j);
    end
    z(1) = z(1)/A0(1,1);
    inftynorm = 0; % z 的无穷范数; 
    pos = 1;       % z 某个分量的模与无穷范数相等对应的坐标
    for index = 1:n
        if abs(z(index)) > inftynorm
            inftynorm = abs(z(index));
            pos = index;
        end
    end
    if inftynorm > z'* x  % 不是最大值继续迭代
        x = zeros(n,1);
        x(pos) = 1;
    else                  % 停止迭代
        norm1 = sum(abs(w));
        OK = 0;
    end
end
        \end{lstlisting}
    \end{subsection}
    \begin{subsection}{估计计算解相对误差}
        \begin{lstlisting}
r = b - A*x;         % 计算残差
rnorm = max(abs(r)); % r 的无穷范数
bnorm = max(abs(b)); % b 的无穷范数
Anorm = max_i sum(abs(A(i,:))); % A 的无穷范数
AInvnorm = InftyNorm_Inv(A);    % A^-1 的无穷范数
error(1,n-4) = rnorm * Anorm * AInvnorm / bnorm;  % 计算解的误差估计
        \end{lstlisting}
    \end{subsection}    
\end{section}

\begin{section}{数值结果及相应分析}
\noindent(1) 估计5-20阶Hilbert矩阵的条件数结果如下:
\begin{table}[htbp!]
    \begin{tabular}{cccccccc}
    \hline
    $n=5$         & $n=6$         & $n=7$         & $n=8$        & $n=9$         & $n=10 $       & $n=11$        & $n=12 $       \\
    9.437e+5  & 2.907e+7  & 9.850e+8  & 3.390e+10 & 1.100e+12 & 3.540e+13 & 1.230e+15 & 3.830e+16 \\ \hline
    $n=13$        & $n=14$        & $n=15$        & $n=16$        & $n=17 $       & $n=18$        & $n=19$        & $n=20$        \\
    4.640e+17 & 1.410e+19 & 1.030e+18 & 1.970e+18 & 1.850e+18 & 9.710e+19 & 3.980e+19 & 3.000e+18\\\hline
    \end{tabular}
    \end{table}

可知: Hilbert矩阵如预期分析,条件数较大,随着阶数增加有着波动上升的趋势.
在$n=20$时达到$10^{18}$,因此可知求解系数矩阵为Hilbert矩阵的线性方程组
是一个非常"病态"的问题,求解的误差很可能会非常大.

\noindent(2)
计算解的误差估计和真实误差结果如图所示(受数字大小影响,不便全部展示在图中.此图只展示$n=21\sim 30$.):
\begin{figure}[htbp!]
    \includegraphics[scale=0.25]{P2_Error_1.jpg}
\end{figure}

可知:

\noindent(a)真实误差与估计误差均大致随$n$增大而增大,偶尔存在突变的位置,可能与向量选取的随机性有关;\\
(b)真实误差始终不大于估计误差(因为是误差的上界);\\
(c)估计误差与真实误差的比例均小于$13.5$,数量级上差距不大,说明估计误差是具有实用性的(即估计误差并不是远大于真实误差).

\end{section}
\end{document}