\documentclass{article}
\usepackage[UTF8]{ctex}
% \usepackage[showframe]{geometry} %调整页边距showframe显示框架
\usepackage{amsmath}  %数学环境
\usepackage{paralist,bbding,pifont} %罗列环境
\usepackage{lmodern}  %中文环境与amsmath格式冲突
\usepackage{array,graphicx}  %插入表格、图片
\usepackage{booktabs}
\usepackage{float}
\usepackage{appendix}
\usepackage{amssymb}
\usepackage{amsthm}
\usepackage{tocloft}  %目录
\usepackage{listings}
\usepackage{xcolor}
\usepackage{hyperref}
\usepackage{setspace}
\usepackage{algorithm}
\usepackage{algpseudocode}
\renewcommand\cftsecdotsep{\cftdotsep}
\renewcommand\cftsecleader{\cftdotfill{\cftsecdotsep}}
\renewcommand {\cftdot}{$ \cdot $}
\renewcommand {\cftdotsep}{1.5}
\hypersetup{colorlinks=true,linkcolor=black}
\usepackage[a4paper, portrait, margin=2.5cm]{geometry}
\renewcommand{\baselinestretch}{1.25} %行间距取多倍行距(设置值为1.5)
\setlength{\baselineskip}{20pt} 

\usepackage{listings}%插入代码
\usepackage{color}
\lstset{%代码格式的配置
extendedchars=false,            % Shutdown no-ASCII compatible
language=Matlab,                % !!!选择代码的语言
basicstyle=\footnotesize\tt,    % the size of the fonts that are used for the code
tabsize=3,                            % sets default tabsize to 3 spaces
numbers=left,                   % where to put the line-numbers
numberstyle=\tiny,              % the size of the fonts that are used for the line-numbers
stepnumber=1,                   % the step between two line-numbers. If it's 1 each line
                                % will be numbered
numbersep=5pt,                  % how far the line-numbers are from the code   %
keywordstyle=\color[rgb]{0,0,1},                % keywords
commentstyle=\color[rgb]{0.133,0.545,0.133},    % comments
stringstyle=\color[rgb]{0.627,0.126,0.941},      % strings
backgroundcolor=\color{white}, % choose the background color. You must add \usepackage{color}
showspaces=false,               % show spaces adding particular underscores
showstringspaces=false,         % underline spaces within strings
showtabs=false,                 % show tabs within strings adding particular underscores
frame=single,                   % adds a frame around the code
captionpos=b,                   % sets the caption-position to bottom
breaklines=true,                % sets automatic line breaking
breakatwhitespace=false,        % sets if automatic breaks should only happen at whitespace
title=\lstname,                 % show the filename of files included with \lstinputlisting;
                                % also try caption instead of title
mathescape=true,escapechar=?    % escape to latex with ?..?
escapeinside={\%*}{*)},         % if you want to add a comment within your code
%columns=fixed,                  % nice spacing
%morestring=[m]',                % strings
%morekeywords={%,...},%          % if you want to add more keywords to the set
%    break,case,catch,continue,elseif,else,end,for,function,global,%
%    if,otherwise,persistent,return,switch,try,while,...},%
 }
%% 页眉
\usepackage{fancyhdr}
\newcommand{\myname}{李佳}
\newcommand{\myid}{2100010793}
\pagestyle{fancy}
\fancyhf{}
\rhead{\myid}
\lhead{\myname}
\cfoot{\thepage}

%%%% Declare %%%
\DeclareMathOperator{\Ran}{Ran}
\DeclareMathOperator{\Dom}{Dom}
\DeclareMathOperator{\Rank}{Rank}

\newcommand{\md}{\mathrm{d}}
\newcommand{\mR}{\mathbb{R}}
\newcommand{\mbF}{\mathbb{F}}
%%% Declare %%%
\newtheorem{innercustomthm}{Problem}
\newenvironment{prob}[1]
{\renewcommand\theinnercustomthm{#1}\innercustomthm}
{\endinnercustomthm}

%%设置
\title{数值代数$\ $第2次上机作业}
\author{李佳~2100010793}
\date{}

%%正文

\begin{document}
\zihao{-4}
\maketitle
\begin{section}{问题描述}
    利用五点差分格式近似求解Dirichlet边界的Poisson方程:
    $$\left\{\begin{aligned}
        -\dfrac{\partial^2u}{\partial x^2}-\dfrac{\partial^2u}{\partial y^2} &= f,\ (x,y)\in\Omega\\
        u &= 0,\ (x,y)\in \partial \Omega
    \end{aligned}\right.$$
    其中区域$\Omega=(0,1)\times(0,1)$,源函数$f = \sin(\pi x)\sin(\pi y)$.
    
    对区域$\Omega$进行均匀剖分,沿$x$轴和$y$轴均以$h = h_x=h_y=\frac{1}{N}$的间距进行等分.记$x_i = (i-1)h_x,y_i = (i-1)h_y$,
    以及$u_{i,j} = u(x_i,y_j),f_{i,j} = f(x_i,y_j),1\leq i,j\leq N+1.$

    当$(x_i,y_j)$为$\Omega$ 内部点时,即$2\leq i,j\leq N$,利用二阶中心差分格式近似$\frac{\partial^2u}{\partial x^2},\frac{\partial^2u}{\partial y^2}:$
    $$\dfrac{\partial^2u}{\partial x^2}\approx \dfrac{u_{i+1,j}+u_{i-1,j}-u_{i,j}}{h_x^2},\dfrac{\partial^2u}{\partial y^2}\approx \dfrac{u_{i,j+1}+u_{i,j-1}-u_{i,j}}{h_y^2},$$
    再由边界条件,知:$$u_{1,j}= u_{N+1,j}=u_{i,1}=u_{i,N+1}= 0,\ 1\leq i,j\leq N+1.$$
    令$U_{i,j}$为$u_{i,j}$的数值近似,可以得到如下方程:
    $$ 4U_{i,j}-U_{i+1,j}-U_{i-1,j}-U_{i,j+1}-U_{i,j-1} = \dfrac{f_{i,j}}{N^2},\ 2\leq i,j\leq N;$$
    $$u_{1,j}= u_{N+1,j}=u_{i,1}=u_{i,N+1}= 0,\ 1\leq i,j\leq N+1.$$
    其中需要求解未知数$U_{i,j},2\leq i,j\leq j$.将$U_{i,j},f_{i,j}$均按列排成向量:
    $$ X = (U_{2,2},U_{2,3},...,U_{2,N},U_{3,2},...,U_{N,N})^T,\ 
    F = (f_{2,2},f_{2,3},...,f_{2,N},f_{3,2},...,f_{N,N})^T,$$
    从而得到线性方程组$$AX=b,$$
    其中$$b = \dfrac{1}{N^2}F,$$
    $$A=\begin{bmatrix}
        M & -I & & &\\
        -I & M & -I & &\\
          & \ddots &\ddots& \ddots & \\
          &  & -I & M & -I\\
          &  &  & -I & M
    \end{bmatrix}_{(N-1)^2\times (N-1)^2},
    M=\begin{bmatrix}
        4 & -1 & & &\\
        -1 & 4 & -1 & &\\
          & \ddots &\ddots& \ddots & \\
          &  & -1 & 4 & -1\\
          &  &  & -1 & 4
    \end{bmatrix}_{(N-1)^2\times (N-1)^2}$$

    对于$N=32,64,128,256,512$时,分别用$LDL^T$方法和带状Guass方法求解
    上述线性方程组,并比较、分析各情况、方法的计算求解时间.
\end{section}
\begin{section}{数值方法}
    \begin{subsection}{一般的$LDL^T$方法}
        我们记矩阵规模$(N-1)^2 = n.$
        注意到矩阵$A$为对称矩阵,顺序主子式均为正,知$A$为对称正定矩阵,可通过$LDL^T$法进行分解,
        再依次求解三角矩阵对应的线性方程组$Lz=b,Dy=z,L^Tx=y$.该方法的工作量为$\frac{1}{3}n^3 =\frac{1}{3}N^6.$ 
    \end{subsection}
    \begin{subsection}{带状Guass方法}
        $A$的顺序主子式均为正,因此对$A$可以直接进行$LU$分解.注意到,矩阵$A$仅在主对角线和上下各$N-1$条对角线的带状区域上有非$0$元素,
        进行一般的Guass消元过程中,只有图示部分的元素可能发生改变.这就是说,在前$n-(N-1)$次Guass消元中,
        不需要对全部的下三角矩阵$A(i+1:n,i+1:n)$操作,只需计算$A(i+1:i+N-1,i+1:i+N-1)$的变化即可.
        这大约可以将工作量压缩为原来的$(\frac{N-1}{n})^2 = \frac{1}{(N-1)^2}$.工作量大约可达到$O(N^4)$.

        \begin{figure}[ht]
            \centering         
            \includegraphics[scale=0.5]{带状Guass消去.jpg}
        \end{figure}

        另一方面, 求解三角矩阵对应的线性方程组$Ly=b,Ux=y$时,通过上述方法求解的$L,U$也仅在
        主对角线和上下各$N-1$条对角线上有非$0$元素,因此求解的代入过程也不需要将向量$b$中的
        全体元素进行更新,(以解$Ly=b$为例)第$i$步($i\leq n-(N-1)$)代入只需计算$b(i+1:i+N-1)$的变化.
        因此每一步三角形方程组求解也可以将原有的工作量$n^2$压缩为原来的$\frac{1}{N-1}$.工作量大约可达到$O(N^3)$.
    \end{subsection}
    \begin{subsection}{$LDL^T$带状消去方法}
        在实际进行$LDL^T$方法求解过程中,发现除对角线附近的带状区域外,其他区域仍然为$0$.事实上,在每一次
        求解$l_{ij},i=j+1,j+2,...,n$的过程中,计算公式为:
            $$ A(j+1:n,j) = (A(j+1:n,j)-A(j+1:n,1:j-1)v(1:j-1))/A(j,j)$$
            $N-1<j\leq n-(N-1)$时,如图所示,矩阵$A(j+1:n,1:j-1)$仅有右上角非$0$,因此作减法、除法的过程仅有$A(j+1:j+N-1,j)$
            发生变化,从而该列只有带状区域内有非$0$元素.由归纳法,易知只有带状区域有非$0$元素.

        \begin{figure}[ht]
            \centering         
            \includegraphics[scale=0.5]{LDL^T带状方法.jpg}
        \end{figure}

        因此,$N-1<j\leq n-(N-1)$时,计算$L$时只需计算$l_{ij},i=j+1,j+2,...,j+N-1$.同理可知(如图)计算$v(i)$时
        只需计算$v(i),i=j-N+1,j-N+2,...,j-1$.这大约也可以将工作量压缩为原来的$\frac{1}{(N-1)^2}$.工作量大约可达到$O(N^4)$.
        解三角形方程组如$2.2$中可进行类似的优化求解.
    \end{subsection}
\end{section}
\begin{section}{理论分析结果}
    从对三种方法的工作量估计来看,问题规模$N$每增大$2$倍,一般的$LDL^T$方法计算时间增加$2^6=64$倍,带状Guass方法和
    $LDL^T$带状方法各增大$2^4=16$倍,一般的$LDL^T$方法与其余两种方法的耗时之比在$N=512$时,与$N=32$时相比将扩大
    $16^2=256$倍.另外,$LDL^T$带状方法与带状Guass方法相比,计算的元素仅在下三角,计算耗时理论上是带状Guass方法的$\frac{1}{2}$.
\end{section}
\newpage
\begin{section}{具体算法实现}
    \begin{subsection}{一般的$LDL^T$方法}
        \begin{lstlisting}
%% Step 1: LDL^T 分解
for j = 1:n
    for i = 1:j-1
        v(i) = A(j,i)*A(i,i);
    end
    A(j,j) = A(j,j) - A(j,1:j-1)*v(1:j-1);
    A(j+1:n,j) = (A(j+1:n,j) - A(j+1:n,1:j-1)*v(1:j-1))/A(j,j);
end

%% Step 2: 解三角形方程组 Lz = b, Dy = z, L^Tx = y
for i = 1:n-1  % 前代法解 Lz = b
    b(i+1:n) = b(i+1:n) - b(i)* A(i+1:n,i);
end
for i = 1: n   % 解 Dy = z
    b(i) = b(i)/ A(i,i);
end
for j = n:-1:2 % 回代法解 L^Tx = y
    b(1:j-1) = b(1:j-1) - b(j)* A(j,1:j-1)';
end
        \end{lstlisting}   
    \end{subsection}
    \begin{subsection}{带状Guass方法}
        \begin{lstlisting}
%% Step 1: 带状 Guass 消去
for i = 1:n-(N-1)     % 前面进行带状高斯消去 
    A(i+1:i+N-1,i) = A(i+1:i+N-1,i)/A(i,i); % 只需计算 L 的(N-1)行
    A(i+1:i+N-1,i+1:i+N-1) = A(i+1:i+N-1,i+1:i+N-1) - A(i+1:i+N-1,i)*A(i,i+1:i+N-1); % 只需更新(N-1)*(N-1)的子阵
end
for i = n-(N-1)+1:n-1 % 后面规模更小 进行一般的高斯消去
    A(i+1:n,i) = A(i+1:n,i)/A(i,i);
    A(i+1:n,i+1:n) = A(i+1:n,i+1:n) - A(i+1:n,i)*A(i,i+1:n);
end

%% Step 2: 解三角形方程组 Ly = b, Ux = y
for i = 1:n-N+1     % 前代法解 Ly = b 此时只需更新向量 b 的(N-1)行
    b(i+1:i+N-1) = b(i+1:i+N-1) - b(i)* A(i+1:i+N-1,i);
end
for i = n-N+2:n-1   % 此时进行正常的前代法求解
    b(i+1:n) = b(i+1:n) - b(i)* A(i+1:n,i);
end
for j = n:-1:N      % 回代法解 Ux = y 此时只需更新向量 b 的(N-1)行
    b(j) = b(j)/A(j,j);
    b(j-N+1:j-1) = b(j-N+1:j-1) - b(j)* A(j-N+1:j-1,j);
end
for j = N-1:-1:2    % 此时进行正常的回代法求解
    b(j) = b(j)/A(j,j);
    b(1:j-1) = b(1:j-1) - b(j)* A(1:j-1,j);
end
b(1) = b(1)/A(1,1);
        \end{lstlisting}
    \end{subsection}
    \begin{subsection}{$LDL^T$带状消去方法}
        \begin{lstlisting}
%% Step 1: LDL^T 带状消去
for j = 1:n
    if j <= N-1          % 前(N-1)列的计算
        for i = 1:j-1
            v(i) = A(j,i)*A(i,i);
        end
        A(j,j) = A(j,j) - A(j,1:j-1)* v(1:j-1);
        % 只需计算 L 在该列中的 (N-1) 行
        % 此处的处理是为尽量运用矩阵向量运算 , 尽管矩阵是上三角
        % 在第 4 节特殊存储方式的处理中 , 只能用标量计算
        % 但充分利用了上三角性质 , 可以减少运算 . 基本过程类似 , 不再赘述
        A(j+1:j+N-1,j) = (A(j+1:j+N-1,j) - A(j+1:j+N-1,1:j-1)*v(1:j-1))/A(j,j); 
    elseif j > n-(N-1)   % 后(N-1)列的计算
        for i = j-N+1:j-1
            v(i-(j-N)) = A(j,i)*A(i,i);  % 向量 v 此时只需计算(N-1)个分量
        end
        A(j,j) = A(j,j) - A(j,j-N+1:j-1) * v(1:N-1);
        A(j+1:n,j) = (A(j+1:n,j) - A(j+1:n,j-N+1:j-1)*v(1:N-1))/A(j,j); 
    else                 % 中间列的计算
        for i = j-N+1:j-1
            v(i-(j-N)) = A(j,i)*A(i,i);  % 向量 v 此时只需计算(N-1)个分量
        end
        A(j,j) = A(j,j) - A(j,j-N+1:j-1)*v(1:N-1);
        A(j+1:j+N-1,j) = (A(j+1:j+N-1,j) - A(j+1:j+N-1,j-N+1:j-1)*v(1:N-1))/A(j,j);
        % 只需计算 L 在该列中的(N-1)行
    end
end

%% Step 2: 解三角形方程组 Lz = b, Dy = z, L^Tx = y
for i = 1: n-N+1 % 前代法解 Lz = b
    b(i+1:i+N-1) = b(i+1:i+N-1) - b(i)* A(i+1:i+N-1,i);  % 此时只需更新向量 b 的(N-1)行
end
for i = n-N+2: n-1
    b(i+1:n) = b(i+1:n) - b(i)* A(i+1:n,i);
end
for i = 1: n     % 解 Dy=z
    b(i) = b(i)/ A(i,i);
end
for j = n: -1: N % 回代法解 L^Tx = y
    b(j-N+1:j-1) = b(j-N+1:j-1) - b(j)* A(j,j-N+1:j-1)'; % 此时只需更新向量 b 的(N-1)行
end
for j = N-1: -1: 2 
    b(1:j-1) = b(1:j-1) - b(j)* A(j,1:j-1)';
end
b(1) = b(1)/A(1,1);
        \end{lstlisting}
    \end{subsection}    
    \begin{subsection}{$N=256,512$时的特殊处理}
        由于$N=256,512$时存储整个矩阵所需的内存过大,因此对矩阵$A$采取了不一样的存储方式(如图).

        \begin{figure}[ht]
            \centering         
            \includegraphics[scale=0.5]{特殊存储方式.jpg}
        \end{figure}

        这种存储方式与Matlab自带函数spdiags输出对角线的方式相似,只存储对角线元素,存储密度更大,因此不会超过内存限度.设这个新矩阵为$B$,
        其中的元素与原矩阵中元素的对应关系是:$$ A(i,j)=B(j,N+j-i),\ \left\lvert i-j\right\rvert \leq N-1.$$
        实际计算中,将原有算法中引用矩阵$A$的地方通过对应关系改成引用$B$中元素即可.

    \end{subsection}
\end{section}

\begin{section}{数值结果及相应分析}
    对每种情况的每种方法,取$5$次计算耗时取均值记入结果
    (除$N=256$时一般的$LDL^T$方法耗时过长,仅进行$2$次计算;
    $N=512$时一般的$LDL^T$方法预计时间超过20小时,未进行计算).
    通过Matlab软件计算得到了如下结果(单位:秒):
    \begin{table}[!ht]
        \centering
        \begin{tabular}{crrr}
        \toprule
            ~ & 一般的$LDL^T$ 方法 & 带状Guass方法 & $LDL^T$带状方法  \\ \midrule
            $N=\ 32$ & 0.4463 & \ \ 0.0182 & \ \ 0.0175 \\ 
            $N=\ 64$ & 38.4076 & \ \ 0.1281 & \ \ 0.0981 \\ 
            $N=128$ & 2442.8544 & \ \ 1.7314 & \ \ 1.3270  \\ 
            $N=256$ & 14346.1470 & 129.3184 & \ 36.0271  \\
            $N=512$ & $\backslash$ & 970.2106 & 513.0896 \\ 
        \bottomrule
        \end{tabular}
    \end{table}

    可以看出,

    \textbf{(1) } 在$N=64,128$时,与小一倍的规模情况相比,一般的$LDL^T$方法计算耗时的增长约为$64$倍,与预估的相差不大;
    $N=256$时采用了稀疏矩阵的方法存储,耗时的增长仅约为$6$倍,个人猜测可能与稀疏矩阵存储
    使得内存占用大幅减少,从而使矩阵元素索引时间的增长比值降低.但尽管如此,$n=512$时的耗时
    也至少需要约$80000$秒,与其他两种方法相比进行了过多的无意义计算,相比来说已不具有实用性.

    \textbf{(2) }除了$N=256$外,带状Guass方法和$LDL^T$带状方法的计算耗时增长均约为$16$倍,与预估的相差不大;
    $N=256$时采用了新的存储方式(4.4节),一定程度上破坏了原矩阵$A$的结构(例如无法直接索引$A$的行向量),
    因此部分矩阵运算只能通过向量运算实现、部分向量运算只能通过标量运算实现,这导致了计算时间的大量增加;
    但由于$LDL^T$带状方法在特殊存储方式中的具体实现中也进行了优化(原本的矩阵-向量运算实际上只需要计算上三角(如2.3节图所示),
    变为标量运算后对这一点进行了优化,减少了一半运算量),因此$LDL^T$方法耗时增长低于$Guass$带状方法.

    \textbf{(3) }带状Guass方法和$LDL^T$带状方法的耗时比例大约为$2:1$,与预估的相差不大.

    \textbf{(4) }$N=256$时,一般的$LDL^T$方法耗时已经达到了带状Guass方法的$106$倍,$LDL^T$带状方法的$384$倍,
    $N=512$时比值将继续增大.由此可见,对稀疏带状矩阵对应的线性方程求解,应尽可能利用带状Guass方法或$LDL^T$带状方法
    之类的能将运算集中在带状区域上的算法,以避免过多的无用计算,获得更高的运行效率.
\end{section}
\end{document}