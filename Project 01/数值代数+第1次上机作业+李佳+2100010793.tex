\documentclass{article}
\usepackage[UTF8]{ctex}
% \usepackage[showframe]{geometry} %调整页边距showframe显示框架
\usepackage{amsmath}  %数学环境
\usepackage{paralist,bbding,pifont} %罗列环境
\usepackage{lmodern}  %中文环境与amsmath格式冲突
\usepackage{array,graphicx}  %插入表格、图片
\usepackage{float}
\usepackage{appendix}
\usepackage{amssymb}
\usepackage{amsthm}
\usepackage{tocloft}  %目录
\usepackage{listings}
\usepackage{xcolor}
\usepackage{hyperref}
\usepackage{setspace}
\usepackage{algorithm}
\usepackage{algpseudocode}
\renewcommand\cftsecdotsep{\cftdotsep}
\renewcommand\cftsecleader{\cftdotfill{\cftsecdotsep}}
\renewcommand {\cftdot}{$ \cdot $}
\renewcommand {\cftdotsep}{1.5}
\hypersetup{colorlinks=true,linkcolor=black}
\usepackage[a4paper, portrait, margin=2.5cm]{geometry}
\renewcommand{\baselinestretch}{1.25} %行间距取多倍行距(设置值为1.5)
\setlength{\baselineskip}{20pt} 

%% 页眉
\usepackage{fancyhdr}
\newcommand{\myname}{李佳}
\newcommand{\myid}{2100010793}
\pagestyle{fancy}
\fancyhf{}
\rhead{\myid}
\lhead{\myname}
\cfoot{\thepage}

%%%% Declare %%%
\DeclareMathOperator{\Ran}{Ran}
\DeclareMathOperator{\Dom}{Dom}
\DeclareMathOperator{\Rank}{Rank}

\newcommand{\md}{\mathrm{d}}
\newcommand{\mR}{\mathbb{R}}
\newcommand{\mbF}{\mathbb{F}}
%%% Declare %%%
\newtheorem{innercustomthm}{Problem}
\newenvironment{prob}[1]
{\renewcommand\theinnercustomthm{#1}\innercustomthm}
{\endinnercustomthm}

%%设置
\title{数值代数$\ $第1次上机作业}
\author{李佳~2100010793}
\date{}

%%正文

\begin{document}
\zihao{-4}
\maketitle
\begin{section}{问题描述}
\noindent 实现Guass消去法计算下列线性方程组,已知精确解是$x=(x_1,...,x_n)^T=(1,...,1)^T,$
    $$\begin{bmatrix} 
        6&1& & & \\
        8&6&1& & \\
         &\ddots&\ddots&\ddots& \\
         & &8&6&1\\
         & & &8&6
    \end{bmatrix}
    \begin{bmatrix}
        x_1\\x_2\\\vdots\\x_{n-1}\\x_n
    \end{bmatrix} = 
    \begin{bmatrix}
        7\\15\\\vdots\\15\\14
    \end{bmatrix}$$
    分别取$n=2,12,24,48,84$,计算$n$取不同值时数值解$x^*$与精确解$x$的误差$\| x^*-x\|_2,\| x^*-x\|_{\infty}.$
    其中$$\| x\|_2 = (\left\lvert x_1\right\rvert^2+...+\left\lvert x_n\right\rvert^2 )^{\frac{1}{2}},
    \ \| x\|_{\infty} = \max_{1\leq i\leq n}\{\left\lvert x_i\right\rvert \}.$$
\end{section}
\begin{section}{数值方法}
    \begin{subsection}{直接三角分解}
        记题目中矩阵为$A$.注意到,$A$的顺序主子式均非$0$,故可进行三角分解.
        将矩阵进行三角分解$A=LU$后,通过前代法解$Ly=b$,再由回代法解$Ux=y$可得数值解$x$.
    \end{subsection}
    \begin{subsection}{列主元Guass消去法}
        对矩阵$A$,在第$k$步Guass消元中先寻找第$k$列的最大元(第$k$至第$n$行之间),找到之后交换最大元
        所在行至第$k$行,再进行Guass消元.

        对矩阵进行列主元三角分解$PA=LU,P=P_{n-1}...P_1$,先计算$b'=Pb=P_{n-1}...P_1b$
        (用向量$u$记录每次交换的行,逐一对向量$b$交换回去),通过前代法解$Ly=b'$,再由回代法解$Ux=y$可得数值解$x$.
    \end{subsection}
    \begin{subsection}{全主元Guass消去法}
        对矩阵$A$,在第$k$步Guass消元中先寻找子矩阵$A(k:n,k:n)$的最大元$A(p,q)$,找到之后交换最大元
        所在行至第$k$行,交换最大元所在列至第$k$列再进行Guass消元.

        对矩阵进行列主元三角分解$PAQ=LU,\ P=P_{n-1}...P_1,\ Q=Q_1Q_2...Q_{n-1}$,先计算$b'=Pb=P_{n-1}...P_1b$
        (用向量$u$记录每次交换的行,逐一对向量$b$交换回去),通过前代法解$Ly=b'$,再由回代法解$Ux'=y$,
        最后计算$x=Qx'=Q_1Q_2...Q_{n-1}x'$(用向量$v$记录每次交换的列,反方向逐一对向量$b$的行交换回去)可得数值解$x$.
    \end{subsection}
\end{section}
\begin{section}{理论分析结果}
    随着矩阵阶数增大,每一步Guass消元、前代法、回代法产生的舍入误差增大,理论上每种方法在两个范数下的误差都会随$n$增大而增大.

    列主元与全主元Guass消去法都可以保证三角分解的下三角矩阵元素$\left\lvert l_{ij}\right\rvert \leq 1 $,
    因此有利于减少舍入误差,理论上在两个范数下,这两种方法与直接三角分解相比误差更小。而全主元相比列主元,理论上误差也更小。

\end{section}
\begin{section}{具体算法实现}
    \begin{subsection}{直接三角分解}
        \small{
            \noindent $\textbf{Step\ 1}:$三角分解

        $ \textbf{for}\ k=1:n-1$ 

        \% 计算Guass变换矩阵,记录在下三角矩阵

        $\qquad A(k+1:n,k) = A(k+1:n,k)/A(k,k)$
         
        \% 计算右下角子矩阵的Guass消元结果

        $\qquad A(k+1:n,k+1:n) = A(k+1:n,k+1:n) - A(k+1:n,k)A(k,k+1:n)$

        $\bf{end}$

        \

        \noindent$\textbf{Step\ 2:}$ 前代法解$Ly=b\ $($y$的结果记录在向量$b$中)

        $\textbf{for}\ j=1:n-1$

        \%\ 将$b(j)$得数代回方程

        $\quad b(j+1:n) = b(j+1:n)-b(j)A(j+1:n,j)$

        $\textbf{end}$

\

        \noindent$\textbf{Step\ 3:}$ 回代法解$Ux=y\ $($x$的结果记录在向量$b$中)

        $\textbf{for}\  j=n:-1:2$

        \%\ 计算$b(j)$

        $\quad b(j) = b(j)/A(j,j)$

        $\quad b(1:j-1) = b(1:j-1)-b(j)A(1:j-1,j)$

        $\textbf{end}$

        \%\ 剩余$b(1)$未计算

        $b(1) = b(1)/A(1,1)$ }
        
    \end{subsection}
    \begin{subsection}{列主元Guass消去法}
        \small{
    \noindent$\textbf{Step\ 1:}$ 列主元三角分解
        
    $\textbf{for}\ k=1:n-1$

        \% 寻找列主元

    $\quad A(p,k) = \max_{k\leq i\leq n}\{\left\lvert A(i,k)\right\rvert \}$

     \% 交换第$k$行与列主元所在的行

    $\quad A(k,1:n)\leftrightarrow A(p,1:n)$

    \% 记录第$k$行与第$p$行交换

    $\quad u(k)=p$

    \% 计算Guass变换矩阵,记录在下三角矩阵

    $\qquad A(k+1:n,k) = A(k+1:n,k)/A(k,k)$
         
    \% 计算右下角子矩阵的Guass消元结果

    $\qquad A(k+1:n,k+1:n) = A(k+1:n,k+1:n) - A(k+1:n,k)A(k,k+1:n)$

    $\bf{end}$

    \ 

    \noindent $\textbf{Step\ 2:}$计算$b'=Pb$

    \%\ $b'=Pb=P_{n-1}...P_1b$,从前往后依次对$b$进行行交换

    $\textbf{for}\ k=1:n-1$

    $\qquad b(k)\leftrightarrow b(u(k))$

    $\textbf{end}$

    \ 

    \noindent $\textbf{Step\ 3:}$前代法解$Ly=b'$(与$4.1$一致)

    \noindent $\textbf{Step\ 4:}$回代法解$Ux=y$(与$4.1$一致)
        }
    \end{subsection}
    \begin{subsection}{全主元Guass消去法}
        \small{
    \noindent$\textbf{Step\ 1:}$ 全主元三角分解
        
    $\textbf{for}\ k=1:n-1$

        \% 寻找全主元

    $\quad A(p,q) = \max_{k\leq i,j\leq n}\{\left\lvert A(i,j)\right\rvert \}$

     \% 交换第$k$行与全主元所在的行、第$k$列与全主元所在的列

    $\quad A(k,1:n)\leftrightarrow A(p,1:n),\ A(1:n,k)\leftrightarrow A(1:n,q)$

    \% 记录第$k$行与第$p$行交换、第$k$列与第$q$列交换

    $\quad u(k)=p,\ v(k)=q$

    \% 计算Guass变换矩阵,记录在下三角矩阵

    $\qquad A(k+1:n,k) = A(k+1:n,k)/A(k,k)$
         
    \% 计算右下角子矩阵的Guass消元结果

    $\qquad A(k+1:n,k+1:n) = A(k+1:n,k+1:n) - A(k+1:n,k)A(k,k+1:n)$

    $\bf{end}$

    \ 

    \noindent $\textbf{Step\ 2:}$计算$b'=Pb$

    \%\ $b'=Pb=P_{n-1}...P_1b$,从前往后依次对$b$进行行交换(与$4.2$一致)

    \ 

    \noindent $\textbf{Step\ 3:}$前代法解$Ly=b'$(与$4.1$一致)

    \noindent $\textbf{Step\ 4:}$回代法解$Ux'=y$(与$4.1$一致)
        
    \noindent $\textbf{Step\ 5:}$计算$x=Qx'$

    \%\ $x=Qx'=Q_1...Q_{n-1}b$,从后往前依次对$x'$进行行交换

    $\textbf{for}\ k=n-1:-1:1$

    $\qquad b(k)\leftrightarrow b(v(k))$ 

    $\textbf{end}$
    }
    \end{subsection}    
    \begin{subsection}{误差计算(每个方法计算完毕后均进行)}
        \small{\%\ 2范数误差计算

        $ error_2 = sqrt(sum_{1\leq i\leq n}(b(i)-1)^2)$

        \ 

        \%\ $\infty$ 范数误差计算

        $ error_{\infty} = \max_{1\leq i\leq n}abs(b(i)-1)$}
    \end{subsection}
\end{section}

\begin{section}{数值结果及相应分析}
    Matlab求解得到以下结果:
    \begin{table}[htbp]
        \centering
        \caption{$2$范数下误差}
        
        \begin{tabular}{cccccc}
        \hline
                   & $n=2$     & $n=12$    & $n=24$    & $n=48$   & $n=84$    \\ \hline
        直接三角分解     & 2.220\ e-16 & 1.537\ e-13 & 6.297\ e-10 & 1.056\ e-02 & 7.259\ e+08  \\
        列主元Guass消去 & 0         & 0         & 0         & 0        & 3.783\ e-06 \\
        全主元Guass消去 & 0         & 0         & 0         & 0        & 3.783\ e-06 \\ \hline
        \end{tabular}
    \end{table}
    \begin{table}[htbp]
        \centering
        \caption{$\infty$范数下误差}

        \begin{tabular}{cccccc}
        \hline
                   & $n=2$     & $n=12$     & $n=24$     & $n=48$    & $n=84$     \\ \hline
        直接三角分解     & 2.220 e-16 & 1.137 e-13 & 4.656 e-10 & 7.812 e-03 & 5.368 e+08  \\
        列主元Guass消去 & 0         & 0          & 0          & 0         & 2.797 e-06 \\
        全主元Guass消去 & 0         & 0          & 0          & 0         & 2.797 e-06 \\ \hline
        \end{tabular}
        \end{table}
        
        可以看出,
        
        \linespread{1.5}\selectfont
        \noindent\textbf{(1)}对每种方法,各范数下的误差随$n$增大而增大.
        
        \noindent\textbf{(2)}对固定的$n$,两种范数下的误差大小相对差距不大,(两种范数均能产生有效数字的情况下)均有$1.000\leq\dfrac{error_2}{error_{\infty}}<1.500$.
        
        \noindent\textbf{(3)}直接三角分解产生的误差在$n$取各值时均大于另外两种方法,且在$n=84$时产生的误差远大于另外两种.
        $n=84$时有$\dfrac{error_2}{\| x^*\|}>\dfrac{error_{\infty}}{\| x^*\|}>5.857\ $e+07,可以认为此时的数值解已经不再可靠.

        \noindent\textbf{(4)}列主元、全主元Guass消去法在$n=2,12,24,48$时的误差都小于$1.000\ $e-16,小于软件中浮点数下界,因此显示为$0$.$\ n=84$时的误差也相对较小,
        $\dfrac{error_{\infty}}{\| x^*\|}<\dfrac{error_2}{\| x^*\|}<4.128\ $e-07,可以认为这两种方法求得的数值解在$n=84$时依然可靠.

        \noindent\textbf{(5)}列主元、全主元Guass消去法在$n=2,12,24,48,84$时产生误差的结果均相等,这是由于问题中的矩阵恰好满足最大元$8$每一列都有,
        且由于矩阵非零元带状分布,Guass消去的过程中只有很少的元素发生了改变,导致寻找子矩阵最大元时恰好寻找的就是当前所在列的列主元,从而导致
        全主元各步求解的结果与列主元一致,因而得到的数值解也相同.
        
        
\end{section}
\end{document}